\documentclass[10pt]{article}

%% Various useful packages and commands from different sources

\usepackage[applemac]{inputenc}
\usepackage[english]{babel}
\usepackage[T1]{fontenc}
\usepackage{cite, url,color} % Citation numbers being automatically sorted and properly "compressed/ranged".
%\usepackage{pgfplots}
\usepackage{graphics,amsfonts}
\usepackage[pdftex]{graphicx}
\usepackage[cmex10]{amsmath}
% Also, note that the amsmath package sets \interdisplaylinepenalty to 10000
% thus preventing page breaks from occurring within multiline equations. Use:
 \interdisplaylinepenalty=2500
% after loading amsmath to restore such page breaks as IEEEtran.cls normally does.

% Compact lists
\usepackage{enumitem}
\usepackage{booktabs}
\usepackage{fancyvrb}

\usepackage{listings} % for Matlab code
\definecolor{commenti}{rgb}{0.13,0.55,0.13}
\definecolor{stringhe}{rgb}{0.63,0.125,0.94}
\lstloadlanguages{Matlab}
\lstset{% general command to set parameter(s)
framexleftmargin=0mm,
frame=single,
keywordstyle = \color{blue},% blue keywords
identifierstyle =, % nothing happens
commentstyle = \color{commenti}, % comments
stringstyle = \ttfamily \color{stringhe}, % typewriter type for strings
showstringspaces = false, % no special string spaces
emph = {for, if, then, else, end},
emphstyle = \color{blue},
firstnumber = 1,
numbers =right, %  show number_line
numberstyle = \tiny, % style of number_line
stepnumber = 5, % one number_line after stepnumber
numbersep = 5pt,
language = {Matlab},
extendedchars = true,
breaklines = true,
breakautoindent = true,
breakindent = 30pt,
basicstyle=\footnotesize\ttfamily
}

\usepackage{array}
% http://www.ctan.org/tex-archive/macros/latex/required/tools/
\usepackage{mdwmath}
\usepackage{mdwtab}
%mdwtab.sty	-- A complete ground-up rewrite of LaTeX's `tabular' and  `array' environments.  Has lots of advantages over
%		   the standard version, and over the version in `array.sty'.
% *** SUBFIGURE PACKAGES ***
\usepackage[tight,footnotesize]{subfigure}
\usepackage[top=2cm, bottom=2cm, right=1.6cm,left=1.6cm]{geometry}
\usepackage{indentfirst}


\setlength\parindent{0pt}
\linespread{1}

\usepackage{mathtools}
\DeclarePairedDelimiter{\ceil}{\lceil}{\rceil}
\DeclarePairedDelimiter{\floor}{\lfloor}{\rfloor}

\begin{document}
\title{Digital Transmission - Homework 1}
\author{Andrea Dittadi, Davide Magrin, Michele Polese}

\maketitle

\section{Estimating the PSD}
In order to estimate the PSD of a given a process $z(k)$ it is necessary to introduce two hypothesis. The first is that the process is at least wide sense stationary. A process $z(k)$ is WSS if $m_z(t) = m_z, \forall t$ and $r_z(t, t - \tau) = r_z(\tau), \forall t$. By looking at the mean and autocorrelation over different non-overlapping windows of 100 samples it appears that the mean is quite constant, as it is the estimate of the real part of the autocorrelation, while the imaginary part of the autocorrelation changes more than the two other estimates. By looking at the sample mean of different windows of 20 samples inside each bigger window the real and imaginary parts of the mean vary very little, despite the great variance that an estimator with few samples has. We conclude that in each window of 100 sample the process can be considered WSS, while it is harder to state that it is WSS in all the 1000 samples. Because of this consideration we will use windows of 100 samples to analyze the PSD of the process and derive conclusions on the presence of the spectral lines.


% Do we agree on this Does it make sense? Can we provide some figures?
% Or we skip these considerations and say that it is wss?

The second hypothesis is that the process is ergodic, and as suggested in \cite{bc} we assume that every WSS process is also ergodic. In this way from a single representation of the process it is possible to estimate the autocorrelation and thus the PSD of the whole process. This is a key assumption for the analysis of the process we are given, without this hypothesis not much can be said. \\



\begin{thebibliography}{10}

\bibitem{bc}
Benvenuto, Cherubini, Algorithms for Communications Systems and their Applications, Wiley, 2004

\end{thebibliography}

\end{document}
